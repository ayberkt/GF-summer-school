\documentclass[]{article}
\usepackage{lmodern}
\usepackage{amssymb,amsmath}
\usepackage{ifxetex,ifluatex}
\usepackage{fixltx2e} % provides \textsubscript
\ifnum 0\ifxetex 1\fi\ifluatex 1\fi=0 % if pdftex
  \usepackage[T1]{fontenc}
  \usepackage[utf8]{inputenc}
\else % if luatex or xelatex
  \ifxetex
    \usepackage{mathspec}
  \else
    \usepackage{fontspec}
  \fi
  \defaultfontfeatures{Ligatures=TeX,Scale=MatchLowercase}
\fi
% use upquote if available, for straight quotes in verbatim environments
\IfFileExists{upquote.sty}{\usepackage{upquote}}{}
% use microtype if available
\IfFileExists{microtype.sty}{%
\usepackage[]{microtype}
\UseMicrotypeSet[protrusion]{basicmath} % disable protrusion for tt fonts
}{}
\PassOptionsToPackage{hyphens}{url} % url is loaded by hyperref
\usepackage[unicode=true]{hyperref}
\hypersetup{
            pdftitle={Day 3 of the Fifth GF Summer School},
            pdfauthor={Ayberk Tosun},
            pdfborder={0 0 0},
            breaklinks=true}
\urlstyle{same}  % don't use monospace font for urls
\usepackage[margin=1in]{geometry}
\IfFileExists{parskip.sty}{%
\usepackage{parskip}
}{% else
\setlength{\parindent}{0pt}
\setlength{\parskip}{6pt plus 2pt minus 1pt}
}
\setlength{\emergencystretch}{3em}  % prevent overfull lines
\providecommand{\tightlist}{%
  \setlength{\itemsep}{0pt}\setlength{\parskip}{0pt}}
\setcounter{secnumdepth}{0}
% Redefines (sub)paragraphs to behave more like sections
\ifx\paragraph\undefined\else
\let\oldparagraph\paragraph
\renewcommand{\paragraph}[1]{\oldparagraph{#1}\mbox{}}
\fi
\ifx\subparagraph\undefined\else
\let\oldsubparagraph\subparagraph
\renewcommand{\subparagraph}[1]{\oldsubparagraph{#1}\mbox{}}
\fi

% set default figure placement to htbp
\makeatletter
\def\fps@figure{htbp}
\makeatother

\usepackage{mathpazo}
\usepackage{eulervm}
\usepackage{ebproof}

\title{Day 3 of the Fifth GF Summer School}
\author{Ayberk Tosun}
\date{}

\begin{document}
\maketitle

\subsection{Resource grammar
libraries}\label{resource-grammar-libraries}

\begin{itemize}
\item
  We want to abstract out the abstract syntax trees commonly used in
  linguistics.

  \begin{itemize}
  \tightlist
  \item
    We do this kind of stuff using resource grammars.
  \end{itemize}
\item
  Here are some of the functions we use for these purposes:

\begin{verbatim}
UseCl : Tense -> Pred -> Cl -> S
PredVP : NP -> VP -> Cl
DetCN : Det -> CN -> NP
ComplV2 : V2 -> NP -> VP
AdjCN : AP -> CN -> CN
PositA : A -> AP
UseN : N -> CN
the_Det : Det
usePron : Pron -> NP
we_Pron : Pron
cat_N : N
\end{verbatim}
\item
  Some of the abstract and concrete syntax looks like the following:

\begin{verbatim}
  cat S; CL; Tense; Pred; VP; NP; CN; Det; AP; Pron; N; A; V2;
    param
      Agr = Ag Gender Number Person
      Case = Nom | Acc
  lincat
    S = {s : Str};
    Cl = {s : PTense => PPol => Str}
    Tense = {t : PTense}
    Pol = {t : PPol}
    VP = {s : PTense => PPol => Agr}
    NP = {s : Case => Str; a : Agr}
    CN = {s : Number => Case => Str; g : Gender}
    AP = {s : Gender => Number => Case => Str}
    Det = {s : Gender => Case => Str; n : Number}
  -- English linearization
  lin
    UseCL t p cl = {s = cl.s ! t.t ! p.p}
    PredVP np vp = {s = \\t, p => np.s ! Nom ++ vp.s ! t ! p ! vp.a}
    AdjCN ap cn = {s = \\n, c => ap.s ! cn.g ! n ! c ++ cn.s ! n ! c; g = cn.g}
    DetCN det cn = {
      s = \\c => det.s ! cn.g ! c ++ cn.s ! det.n ! c;
      a = Ag cn.g det.n P3
    }
\end{verbatim}
\item
  \textbf{API}

\begin{verbatim}
mkNP : Det -> CN -> NP
mkNP : Pron -> NP
\end{verbatim}
\item
  \textbf{Assignment}
\end{itemize}

write a concrete module for the City grammar in your chosen language
using the RGL API.

\begin{itemize}
\tightlist
\item
  Documentation can be found in
  \href{http://www.grammaticalframework.org/lib/doc/synopsis.html}{GF
  Resource Grammar Library: Synopsis}.
\end{itemize}

\end{document}
